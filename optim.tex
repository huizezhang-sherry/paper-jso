% Options for packages loaded elsewhere
\PassOptionsToPackage{unicode}{hyperref}
\PassOptionsToPackage{hyphens}{url}
\PassOptionsToPackage{dvipsnames,svgnames,x11names}{xcolor}
%
\documentclass[
  number,
  preprint,
  3p]{elsarticle}

\usepackage{amsmath,amssymb}
\usepackage{iftex}
\ifPDFTeX
  \usepackage[T1]{fontenc}
  \usepackage[utf8]{inputenc}
  \usepackage{textcomp} % provide euro and other symbols
\else % if luatex or xetex
  \usepackage{unicode-math}
  \defaultfontfeatures{Scale=MatchLowercase}
  \defaultfontfeatures[\rmfamily]{Ligatures=TeX,Scale=1}
\fi
\usepackage{lmodern}
\ifPDFTeX\else  
    % xetex/luatex font selection
\fi
% Use upquote if available, for straight quotes in verbatim environments
\IfFileExists{upquote.sty}{\usepackage{upquote}}{}
\IfFileExists{microtype.sty}{% use microtype if available
  \usepackage[]{microtype}
  \UseMicrotypeSet[protrusion]{basicmath} % disable protrusion for tt fonts
}{}
\makeatletter
\@ifundefined{KOMAClassName}{% if non-KOMA class
  \IfFileExists{parskip.sty}{%
    \usepackage{parskip}
  }{% else
    \setlength{\parindent}{0pt}
    \setlength{\parskip}{6pt plus 2pt minus 1pt}}
}{% if KOMA class
  \KOMAoptions{parskip=half}}
\makeatother
\usepackage{xcolor}
\setlength{\emergencystretch}{3em} % prevent overfull lines
\setcounter{secnumdepth}{5}
% Make \paragraph and \subparagraph free-standing
\ifx\paragraph\undefined\else
  \let\oldparagraph\paragraph
  \renewcommand{\paragraph}[1]{\oldparagraph{#1}\mbox{}}
\fi
\ifx\subparagraph\undefined\else
  \let\oldsubparagraph\subparagraph
  \renewcommand{\subparagraph}[1]{\oldsubparagraph{#1}\mbox{}}
\fi


\providecommand{\tightlist}{%
  \setlength{\itemsep}{0pt}\setlength{\parskip}{0pt}}\usepackage{longtable,booktabs,array}
\usepackage{calc} % for calculating minipage widths
% Correct order of tables after \paragraph or \subparagraph
\usepackage{etoolbox}
\makeatletter
\patchcmd\longtable{\par}{\if@noskipsec\mbox{}\fi\par}{}{}
\makeatother
% Allow footnotes in longtable head/foot
\IfFileExists{footnotehyper.sty}{\usepackage{footnotehyper}}{\usepackage{footnote}}
\makesavenoteenv{longtable}
\usepackage{graphicx}
\makeatletter
\def\maxwidth{\ifdim\Gin@nat@width>\linewidth\linewidth\else\Gin@nat@width\fi}
\def\maxheight{\ifdim\Gin@nat@height>\textheight\textheight\else\Gin@nat@height\fi}
\makeatother
% Scale images if necessary, so that they will not overflow the page
% margins by default, and it is still possible to overwrite the defaults
% using explicit options in \includegraphics[width, height, ...]{}
\setkeys{Gin}{width=\maxwidth,height=\maxheight,keepaspectratio}
% Set default figure placement to htbp
\makeatletter
\def\fps@figure{htbp}
\makeatother

\makeatletter
\makeatother
\makeatletter
\makeatother
\makeatletter
\@ifpackageloaded{caption}{}{\usepackage{caption}}
\AtBeginDocument{%
\ifdefined\contentsname
  \renewcommand*\contentsname{Table of contents}
\else
  \newcommand\contentsname{Table of contents}
\fi
\ifdefined\listfigurename
  \renewcommand*\listfigurename{List of Figures}
\else
  \newcommand\listfigurename{List of Figures}
\fi
\ifdefined\listtablename
  \renewcommand*\listtablename{List of Tables}
\else
  \newcommand\listtablename{List of Tables}
\fi
\ifdefined\figurename
  \renewcommand*\figurename{Figure}
\else
  \newcommand\figurename{Figure}
\fi
\ifdefined\tablename
  \renewcommand*\tablename{Table}
\else
  \newcommand\tablename{Table}
\fi
}
\@ifpackageloaded{float}{}{\usepackage{float}}
\floatstyle{ruled}
\@ifundefined{c@chapter}{\newfloat{codelisting}{h}{lop}}{\newfloat{codelisting}{h}{lop}[chapter]}
\floatname{codelisting}{Listing}
\newcommand*\listoflistings{\listof{codelisting}{List of Listings}}
\makeatother
\makeatletter
\@ifpackageloaded{caption}{}{\usepackage{caption}}
\@ifpackageloaded{subcaption}{}{\usepackage{subcaption}}
\makeatother
\makeatletter
\@ifpackageloaded{tcolorbox}{}{\usepackage[skins,breakable]{tcolorbox}}
\makeatother
\makeatletter
\@ifundefined{shadecolor}{\definecolor{shadecolor}{rgb}{.97, .97, .97}}
\makeatother
\makeatletter
\makeatother
\makeatletter
\makeatother
\journal{Journal of Multivariate Analysis}
\ifLuaTeX
  \usepackage{selnolig}  % disable illegal ligatures
\fi
\usepackage[]{natbib}
\bibliographystyle{elsarticle-num-names}
\IfFileExists{bookmark.sty}{\usepackage{bookmark}}{\usepackage{hyperref}}
\IfFileExists{xurl.sty}{\usepackage{xurl}}{} % add URL line breaks if available
\urlstyle{same} % disable monospaced font for URLs
\hypersetup{
  pdftitle={Performance of Jellyfish Search Optimiser on Projection Pursuit Problems},
  pdfauthor={Alice Anonymous; Bob Security; Cat Memes; Derek Zoolander},
  pdfkeywords={projection pursuit, optimization, jellyfish optimiser},
  colorlinks=true,
  linkcolor={blue},
  filecolor={Maroon},
  citecolor={Blue},
  urlcolor={Blue},
  pdfcreator={LaTeX via pandoc}}

\setlength{\parindent}{6pt}
\begin{document}

\begin{frontmatter}
\title{Performance of Jellyfish Search Optimiser on Projection Pursuit
Problems}
\author[1]{Alice Anonymous%
\corref{cor1}%
}
 \ead{alice@example.com} 
\author[2]{Bob Security%
%
}
 \ead{bob@example.com} 
\author[2]{Cat Memes%
%
}
 \ead{cat@example.com} 
\author[]{Derek Zoolander%
%
}
 \ead{derek@example.com} 

\affiliation[1]{organization={Some Institute of Technology, Department
Name},addressline={Street Address},city={City},postcode={Postal
Code},postcodesep={}}
\affiliation[2]{organization={Another University, Department
Name},addressline={Street Address},city={City},postcode={Postal
Code},postcodesep={}}

\cortext[cor1]{Corresponding author}




        
\begin{abstract}
This is the abstract. Lorem ipsum dolor sit amet, consectetur adipiscing
elit. Vestibulum augue turpis, dictum non malesuada a, volutpat eget
velit. Nam placerat turpis purus, eu tristique ex tincidunt et. Mauris
sed augue eget turpis ultrices tincidunt. Sed et mi in leo porta
egestas. Aliquam non laoreet velit. Nunc quis ex vitae eros aliquet
auctor nec ac libero. Duis laoreet sapien eu mi luctus, in bibendum leo
molestie. Sed hendrerit diam diam, ac dapibus nisl volutpat vitae.
Aliquam bibendum varius libero, eu efficitur justo rutrum at. Sed at
tempus elit.
\end{abstract}





\begin{keyword}
    projection pursuit \sep optimization \sep 
    jellyfish optimiser
\end{keyword}
\end{frontmatter}
    \ifdefined\Shaded\renewenvironment{Shaded}{\begin{tcolorbox}[breakable, enhanced, boxrule=0pt, frame hidden, borderline west={3pt}{0pt}{shadecolor}, interior hidden, sharp corners]}{\end{tcolorbox}}\fi

\emph{Let's use British English (``American or British usage is
accepted, but not a mixture of these'')}

\hypertarget{introduction-nicolas-and-jessica}{%
\section{Introduction {[}Nicolas and
Jessica{]}}\label{introduction-nicolas-and-jessica}}

The artificial jellyfish search (JS) algorithm \citep{chou_novel_2021}
is a swarm-based metaheuristic optimisation algorithm inspired by the
search behaviour of jellyfish in the ocean. It is one of the newest
swarm intelligence algorithms \citep{rajwar_exhaustive_2023}, which was
shown to have stronger search ability and faster convergence with few
algorithmic parameters compared to classic optimization methods
\citep{chou_novel_2021}-\citep{chou_recent_2022}.

The rest of the paper is organised as follows:
Section~\ref{sec-background} introduces the projection pursuit method,
including the indexes function and optimisation.
Section~\ref{sec-theory} introduces the jellyfish optimiser and proposes
mathematical expressions to measure the . Section~\ref{sec-simulation}
applies the jellyfish optimiser through different projection pursuit
problems with varying dimensions and index functions.
Section~\ref{sec-conclusion} concludes the paper.

\hypertarget{sec-background}{%
\section{Projection pursuit, index functions and optimisation {[}Di and
Sherry{]}}\label{sec-background}}

\hypertarget{sec-theory}{%
\section{The jellyfish optimiser and property for good optimisers
{[}Nicolas and Jessica{]}}\label{sec-theory}}

The jellyfish optimiser \citep{chou_novel_2021} \ldots{}

\citet{laa_using_2020} has proposed five criteria for assessing
projection pursuit indexes (smoothness, squintability, flexibility,
rotation invariance, and speed). Since not all the properties affects
the optimisation, here we define the three relevant properties
(smoothness, squintability, and speed) mathematically and show that the
jellyfish optimiser \ldots{}

\hypertarget{sec-simulation}{%
\section{Application {[}Di and Sherry{]}}\label{sec-simulation}}

The jellyfish optimiser has been implemented in the tourr package
\citep{wickham_tourr_2011} and we will use the diagnostic plots proposed
in the ferrn package \citep{RJ-2021-105} to visualise the optimisation
process.

\hypertarget{going-beyond-10d}{%
\subsection{Going beyond 10D}\label{going-beyond-10d}}

The pipe-finding problem is initially used to investigate indexes and
optimisers in \citet{laa_using_2020}, and we extend it from a 6D problem
to a 12D problem.

Jellyfish optimiser, as a multi-start algorithm, is efficient in
{[}\ldots{]} for high-dimensional problems

\begin{figure}

{\centering \includegraphics{optim_files/figure-pdf/unnamed-chunk-2-1.pdf}

}

\caption{sthis sdfaksdlf}

\end{figure}

\begin{figure}

{\centering \includegraphics{optim_files/figure-pdf/unnamed-chunk-3-1.pdf}

}

\caption{sthis sdfaksdlf}

\end{figure}

\hypertarget{on-skewness-and-kurtosis-index}{%
\subsection{On skewness and kurtosis
index}\label{on-skewness-and-kurtosis-index}}

\hypertarget{another-data-example}{%
\subsection{Another data example}\label{another-data-example}}

\hypertarget{sec-conclusion}{%
\section{Conclusion {[}Di and Sherry{]}}\label{sec-conclusion}}


\renewcommand\refname{References}
  \bibliography{bibliography.bib}


\end{document}
